%Pakete;
%A4, Report, 12pt
\documentclass[ngerman,a4paper,12pt]{scrreprt}
\usepackage[a4paper, right=20mm, left=20mm,top=20mm, bottom=30mm, marginparsep=5mm, marginparwidth=5mm, headheight=7mm, headsep=15mm,footskip=15mm]{geometry}

%Papierausrichtungen
\usepackage{pdflscape}
\usepackage{lscape}

%Deutsche Umlaute, Schriftart, Deutsche Bezeichnungen
\usepackage[utf8]{inputenc}
\usepackage[T1]{fontenc}
\usepackage[ngerman]{babel}

%quellcode
\usepackage{listings}

%tabellen
\usepackage{tabularx}

%listen und aufzählungen
\usepackage{paralist}

%farben
\usepackage[svgnames,table,hyperref]{xcolor}

%symbole
\usepackage{latexsym,textcomp}

%font
\usepackage{helvet}
\renewcommand{\familydefault}{\sfdefault}

%Abkürzungsverzeichnisse
\usepackage[printonlyused]{acronym}

%Bilder
\usepackage{graphicx} %Bilder
\usepackage{float}	  %"Floating" Objects, Bilder, Tabellen...
\usepackage[space]{grffile} %Leerzechen Problem bei includegraphics
\usepackage{wallpaper} %Seitenhintergrund setzen
\usepackage{transparent} %Transparenz

%for
\usepackage{forloop}
\usepackage{ifthen}

%Dokumenteigenschaften
\title{Repetitionsfragen Dbs2}
\author{Tobias Blaser}
\date{\today{}, Rapperswil}


%Kopf- /Fusszeile
\usepackage{fancyhdr}
\usepackage{lastpage}

\pagestyle{fancy}
	\fancyhf{} %alle Kopf- und Fußzeilenfelder bereinigen
	\renewcommand{\headrulewidth}{0pt} %obere Trennlinie
	\fancyfoot[L]{Seite \thepage/\pageref{LastPage}} %Fusszeile mitte
	\fancyfoot[R]{\today{}} %Fusszeile rechts
	\renewcommand{\footrulewidth}{0.4pt} %untere Trennlinie

%Kopf-/ Fusszeile auf chapter page
\fancypagestyle{plain} {
	\fancyhf{} %alle Kopf- und Fußzeilenfelder bereinigen
	\renewcommand{\headrulewidth}{0pt} %obere Trennlinie
	\fancyfoot[L]{Seite \thepage/\pageref{LastPage}} %Fusszeile mitte
	\fancyfoot[R]{\today{}} %Fusszeile rechts
	\renewcommand{\footrulewidth}{0.4pt} %untere Trennlinie
}

\usepackage{changepage}

% Abkürzungen für Kapitel, Titel und Listen
\input{commands/shortcutsListAndChapter}
\input{commands/TextStructuringBoxes}

%links, verlinktes Inhaltsverzeichnis, PDF Inhaltsverzeichnis
\usepackage[bookmarks=true,
bookmarksopen=true,
bookmarksnumbered=true,
breaklinks=true,
colorlinks=true,
linkcolor=black,
anchorcolor=black,
citecolor=black,
filecolor=black,
menucolor=black,
pagecolor=black,
urlcolor=black
]{hyperref} % Paket muss unbedingt als letzes eingebunden werden!

\usepackage{graphicx}
\begin{document}

% Inhaltsverzeichnis
\tableofcontents
\clearpage

\ch{Repetition Dbs1}
\ol
	\li Erklären Sie das ANSI 3-Ebenen Modell
	\li Erklären SIe OLTP
	\li Nennen Sie die vier ACID Kriterien. Wozu werden Sie verwendet?
	\li Lesen Sie mit einer group by Anfrage aus einer Liste mit Absolventen die Studiengangsnotendurchschnitte heraus.
\olS

\ch{Oracle PL/SQL}
\olR
	\li Was ist der Unterschied zwischen einer 'Stored Procedure' und einer Funktion?
	\li Wie ist eine Funktion aufgebaut, wie eine Stored Procedure? Wie führen Sie den Code jeweils aus (Einbetten in anderen Code?)?
	\li Wie funktioniert in PL/SQL das Exception Handling? Was sind unbenannte/benannte Benutzerexceptions, was benannte Systemexceptions?
	\li Für welche drei Anwendungszwecke setzen Sie 'Stored Procedures' ein?
	\li Welchen Vorteil bieten Stored Procedures in Zusammenhang mit Views?
	\li Wozu dient die Pseudotabelle 'DUAL'?
\olS

\ch{Stored Procedures}
\olR
	\li Erklären SIe den Unterschied zwischen einem anonymen PL/SQL Block und einer Stored Procedure. Nennen Sie je Vor- und Nachteile.
	\li Wie funktionieren Stored Procedures mit Java?
	\li Wie wird der Java Code verarbeitet und wo wird er gespeichert?
	\li Beschreiben Sie, was sie alles unternehmen müssen, um eine Java Stored Procedure zum Laufen zu bekommen.
	\li Machen Sie ein Beispiel, wie sie Stored Procedures mit Python verwenden.
\olS

\ch{Packages}
\olR
	\li Was sind PL/SQL Packages, wozu können Sie eingesetzt werden?
	\li Warum besitzt ein DBS keine gewöhliche Input/Output Shell? Wie können Sie trotzdem Angaben von einer Kommandozeile einlesen und ausgeben?
	\li Machen Sie eine Beispiel für ein Package
\olS

\ch{Cursors}
\olR
	\li Wozu verwenden Sie Cursors?
	\li Lesen Sie mittels eines Cursors eine Tabelle mit Standorten und Temperaturen aus. Ist die Temperatur höher als x Grad, so übertragen Sie die Datenbankzeile in eine neue Tabelle.
	Übergeben Sie x als Parameter, sodass der Cursor von aussen gesteuert werden kann.
	\li Was können Sie mit den vier Curso-Attributen machen?
\olS

\end{document}
